% Created 2014-10-12 日 16:05
\documentclass{ltjsarticle}
\usepackage[kozuka-pr6n]{luatexja-preset}
\usepackage[left=2truecm,right=2truecm,top=3truecm,bottom=3truecm]{geometry}
\usepackage[utf8]{inputenc}
\usepackage[T1]{fontenc}
\usepackage{fixltx2e}
\usepackage{graphicx}
\usepackage{longtable}
\usepackage{float}
\usepackage{wrapfig}
\usepackage{rotating}
\usepackage[normalem]{ulem}
\usepackage{amsmath}
\usepackage{textcomp}
\usepackage{marvosym}
\usepackage{wasysym}
\usepackage{amssymb}
\usepackage{hyperref}
\tolerance=1000
\author{井上優作}
\date{\today}
\title{ソースコード講読 alter.c}
\hypersetup{
  pdfkeywords={},
  pdfsubject={},
  pdfcreator={Emacs 24.3.1 (Org mode 8.2.8)}}
\begin{document}

\maketitle

\section*{ALTER TABLEについて}
\label{sec-1}

SQLite3の \verb~ALTER TABLE~ コマンドは、テーブルの名前を変更するか、またはカラムを追加することができるコマンドである。以下のようにして実行する。

\begin{verbatim}
ALTER TABLE tbl RENAME TO new_table_name
ALTER TABLE tbl ADD COLUMN column-def
\end{verbatim}

\section*{ソースコード}
\label{sec-2}

\begin{itemize}
\item L21: このファイルのコードは、ビルド時に \verb~ALTER TABLE~ コマンドを除かない限り存在する。
\end{itemize}

\subsection*{LL.37-84 static void renameTableFunc()}
\label{sec-2-1}

コメント中にあるコード例の引数が、この関数の第3引数に配列( \verb~**argv~ )となって入っている。42-43行目で、SQL文とテーブルの名前が取り出されている。テーブル名の特定規則は、 \verb~TK_LP~ または \verb~TK_USING~ の後に続く最初のスペースでないトークンであるとしており、これを59−78行目で実行している。この関数自体は返り値は \verb~void~ だが、80行目の \verb~zRet~ が実質的な返り値? 

\subsection*{LL.102-148 static void renameParentFunc()}
\label{sec-2-2}

外部キー制約の定義を修正するための関数。120−127行目で古い名前を探して、129−138行目でそれを新しい名前に変更しているような気がする。
% Emacs 24.3.1 (Org mode 8.2.8)
\end{document}
