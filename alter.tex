% Created 2014-10-12 日 23:03
\documentclass{ltjsarticle}
\usepackage[kozuka-pr6n]{luatexja-preset}
\usepackage[left=2truecm,right=2truecm,top=3truecm,bottom=3truecm]{geometry}
\usepackage[utf8]{inputenc}
\usepackage[T1]{fontenc}
\usepackage{fixltx2e}
\usepackage{graphicx}
\usepackage{longtable}
\usepackage{float}
\usepackage{wrapfig}
\usepackage{rotating}
\usepackage[normalem]{ulem}
\usepackage{amsmath}
\usepackage{textcomp}
\usepackage{marvosym}
\usepackage{wasysym}
\usepackage{amssymb}
\usepackage{hyperref}
\tolerance=1000
\author{井上優作}
\date{\today}
\title{ソースコード講読 alter.c}
\hypersetup{
  pdfkeywords={},
  pdfsubject={},
  pdfcreator={Emacs 25.0.50.1 (Org mode 8.2.8)}}
\begin{document}

\maketitle

\section*{ALTER TABLEについて}
\label{sec-1}

SQLite3の \verb~ALTER TABLE~ コマンドは、テーブルの名前を変更するか、またはカラムを追加することができるコマンドである。以下のようにして実行する。

\begin{verbatim}
ALTER TABLE tbl RENAME TO new_table_name
ALTER TABLE tbl ADD COLUMN column-def
\end{verbatim}

\section*{ソースコード}
\label{sec-2}

\begin{itemize}
\item L21: このファイルのコードは、ビルド時に \verb~ALTER TABLE~ コマンドを除かない限り存在する。
\end{itemize}

\subsection*{LL.37-84 static void renameTableFunc}
\label{sec-2-1}

コメント中にあるコード例の引数が、この関数の第3引数に配列( \verb~**argv~ )となって入っている。42-43行目で、SQL文とテーブルの名前が取り出されている。テーブル名の特定規則は、 \verb~TK_LP~ または \verb~TK_USING~ の後に続く最初のスペースでないトークンであるとしており、これを59−78行目で実行している。この関数自体は返り値は \verb~void~ だが、80行目の \verb~zRet~ が実質的な返り値? 

\subsection*{LL.102-148 static void renameParentFunc}
\label{sec-2-2}

外部キー制約の定義を修正するための関数。120−127行目で古い名前を探して、129−138行目でそれを新しい名前に変更しているような気がする。

\subsection*{LL.158-224 static void renameTriggerFunc}
\label{sec-2-3}

第1引数が \verb~CREATE TRIGGER~ 文,第2引数がテーブルの名前で,これが第3引数のテーブル名に置き換えられる. \verb~CREATE TRIGGER~ 文の中のテーブル名は \verb~TK_ON~ か \verb~TK_DOT~ と \verb~TK_WHEN~, \verb~TK_BEGIN~ または \verb~TK_FOR~ の間にあるトークンである.この規則に従って,181-215行目でテーブルの名前を特定している.今までに出てきた関数と違って,古い名前を新しい名前に置き換えているようなコードが見つからない.

\subsection*{LL.230-247 void sqlite3AlterFunctions}
\label{sec-2-4}

今までに定義した関数をグローバルな関数定義ハッシュテーブルに登録しているらしいということだけ分かった(L245の \verb~sqlite3FuncDefInsert~ は \verb~callback.c~:L304 で定義されている).

\subsection*{LL.265-274 static char *whereOrName}
\label{sec-2-5}

複数のconstantを受け取って \verb~name=<constant1> OR name=<constant2> OR ...~ というフォーマットの文字列を返す関数.フォーマット指定子に使われている \verb~%Q~ は標準で実装されているものではない? また,270行目だけ見ると,3つ以上のconstantがあるときにどうやって \verb~OR~ でつないだ文字列を返すのかが分からない.

\subsection*{LL.283-290 char *whereForeignKeys}
\label{sec-2-6}

上で定義した関数を使ってwhere句を返す関数. \verb~zWhere~ という1つの変数に代入し続けている. \verb~*whereOnName~ の返し方が再帰的になっている?

\subsection*{LL.299-323 static char *whereTempTriggers}
\label{sec-2-7}

全てのテンポラリートリガーを選択するwhere句文字列を生成するための関数.309-316行目でそれを生成して,317-322行目で返している?

\subsection*{LL.333-372 void reloadTableSchema}
\label{sec-2-8}

テーブルの内部表現を一旦削除して読み込み直す関数.341-345行でVDBEとデータベースを準備,347-353行目で内部スキーマからトリガーを削除,357行目でテーブルとインデックスを削除,360−362行目でテーブルを再読み込みしている.368-370行目では,もしテーブルがテンポラリーデータベースに無ければテンポラリートリガーを読み込む(?).
% Emacs 25.0.50.1 (Org mode 8.2.8)
\end{document}
